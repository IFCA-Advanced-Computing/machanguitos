\documentclass{beamer}
\usepackage{lmodern}
\usepackage{listings}
\usepackage[utf8]{inputenc}

\definecolor{kugreen}{RGB}{50,93,61}
\definecolor{mygreen}{rgb}{0,0.4,0}
\definecolor{gray97}{gray}{.97}
\definecolor{gray75}{gray}{.75}

\usetheme{PaloAlto}

\usecolortheme[named=kugreen]{structure}
\setbeamertemplate{navigation symbols}{}

\lstdefinestyle{agentlang}{language=C++,
  frame=tLBr,
  framerule=0.2pt,
  backgroundcolor=\color{gray97},
  captionpos=b,
  %
  numbers=left,
  stepnumber=1,
  numberstyle=\tiny,
  numbersep=7pt,
  numberfirstline=false,
  %
  extendedchars=true,
  basicstyle=\scriptsize\ttfamily,
  commentstyle=\color{mygreen}\ttfamily,
  stringstyle=\ttfamily\color{magenta},
  showstringspaces=false,
  keywordstyle=\color{violet}\bfseries,
  morekeywords={end,function,config},
  morecomment=[l]{\-\-},
  breaklines=true,
}

\lstdefinestyle{console}{
  frame=l,
  backgroundcolor=\color{gray75},
  basicstyle=\scriptsize\bf\ttfamily,
}

\title{Machanguitos}
\subtitle{The Simpliest Multi-Agent System ever}
\author{Luis Cabellos}
\date{\today}

\begin{document}

\frame{\titlepage}

%% \frame{\tableofcontents}

\section{Introduction}

\frame{
  Current Solutions:
  \begin{itemize}
    \item Ugly Frameworks\footnote{There is something ugly that C++: Fortran.}
    \item Doesn't Scale
  \end{itemize}
  \pause
  \frametitle{Introduction}
  \begin{block}{Machanguitos}
    The Simplest Multi-Agent System Ever.
    
    Easy, Scalable, but Powerful.
  \end{block}
  
}

\section{Features}

\begin{frame}[fragile]
  \frametitle{Machen Applications}
  \begin{block}{}
    Machanguitos is not a Framework.
  \end{block}
  \pause
  \begin{block}{}
    Machanguitos is an application that simulate agents.
  \end{block}
  \pause
  \begin{block}{}
    Agents are defined using Lua Language.
  \end{block}
\begin{lstlisting}[style=agentlang]
function factorial(n)
  if n == 0 then
    return 1
  else
    return n * factorial(n - 1)
  end
end
\end{lstlisting}
\end{frame}

\frame{
  \frametitle{Machanguitos Scalability}
  \begin{itemize}
    \item Enviroment: NoSQL database
    \item MPI for use multiple computing resources
    \item Agents comunicate only with the Enviroment
  \end{itemize}
}

\section{Example}

\begin{frame}[fragile]
  \frametitle{Simple Example: Agent}
\begin{lstlisting}[style=agentlang,caption=cow.lua]
function Agent:init()
   self.x = math.random( 10 );
   self.y = math.random( 10 );
end

function Agent:update( delta )
   self.x = self.x + 0.1*delta;
   self.y = self.y + 0.1*delta;
   io.write( "cow pos = ("..self.x..","..self.y..")\n" )
end
\end{lstlisting}
\end{frame}

\begin{frame}[fragile]
  \frametitle{Simple Example: Main}
\begin{lstlisting}[style=agentlang,caption=ex01.lua]
-- io.write( "Cow Example\n" );
config.add_agent( "cow", 10 );

config.setvars( {
  starttime = 0.0,
  endtime = 10.0,
  iters = 10,
} );
\end{lstlisting}
  
\begin{lstlisting}[style=console]
# machen ~uc00103/agents/ex01.lua
\end{lstlisting}
\end{frame}

\begin{frame}[fragile]
  \frametitle{Simple Example: Saving Output}
\begin{lstlisting}[style=agentlang,caption=cow.lua]
AgentClass:save( 'x', 'y' )

function Agent:init()
   self.x = math.random( 10 );
   self.y = math.random( 10 );
end

function Agent:update( delta )
   self.x = self.x + 0.1*delta;
   self.y = self.y + 0.1*delta;
   io.write( "cow pos = ("..self.x..","..self.y..")\n" )
end
\end{lstlisting}
\end{frame}

\end{document}
